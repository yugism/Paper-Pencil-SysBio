\chapter{まえがき}
``... there are many truths of which the full meaning cannot be realized, until personal experience has brought it home.'' (John Stuart Mill ``On Liberty'')

名詞が複数形であることを強調したい箇所では、名詞に「[たち]」を付けました。

読者層

高校生
 高校(+大学)で習う数学を生かせる先端生命科学がある、と聞いて興味がわいた高校生

生物系学科の大学生
 大学院進学先として、実験だけでなく数理的方法も使うラボに興味があるけれど何を勉強したらいいかわからない人の手助けに

システム生物学的方法に関心がある大学院生
 有力な手法らしいけれど何をしていいかわからない。特に数学っぽいところがわからない。
 でもコンピュータ使うし難しそう。と、敬遠しているのは損です。 

すでにシステム生物学と接点のある大学院生
 コンピュータ上にパスウェイの模型をつくって動かすだけでは、背景にある理論を理解せずに使ってしまいます。紙と鉛筆でできる演習問題を解くことで「腑に落とす」 ための教材を目指しました。

これらの人々に共通
・ミカエリス・メンテン式の導き方、行列計算のイロハから始めるので大丈夫!
・生物のことを例題に、今まで高校まで(一部は大学初年次まで)で学んだ数学がどう生物に役立つかわかります。
・諸分野のプライマーを用意すること。「伸長反応」はfurther readings1を示すので個人に任せる。→ より上級の本へのプライマーになればOK

生命科学への情熱を動因として、システム生物学で使う数学を身につける。


あるいはご自分の分野にこういったキーワードが入ってきて取り入れてみたい方:



システム生物学。数理モデル。シミュレーション。多因子。

こういったキーワードに惹かれた方(若い人):

代謝が進む速さについては、20世紀初頭のミカエリス(Michaelis)とメンテ
ン(Menten)の研究に始まる酵素反応速度論(enzyme kinetics)により、数式化する方法
が確立されてきました。そのため、数理モデルの発展はシグナル伝達や遺伝子発現に先立って代謝系で進みました。



システム生物学に必要となる数理の初歩を一冊で済ませることを目指しました。



暗黙知にコンパイルするための演習重視。

式の導出を重視しました。

必要な知識は変数分離法と3x3の逆行列。あと偏微分少々。

講義を行う場合の実施案も示しました。

融合領域では個別分野を効率よく学ぶ必要があります。そのためには教育法も進歩しなければならない。「自分が苦労したようにあなたも苦労しなさい」というのではまるで「虐待の連鎖」ではありませんか。


